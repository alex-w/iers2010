\documentclass[12pt, a4paper, twoside]{article}

\usepackage{graphicx}
\usepackage{xargs} %% Use more than one optional parameter in a new command
\usepackage{caption}
\captionsetup[table]{position=bottom}
\usepackage{hyperref}
\hypersetup{
  colorlinks=true,
  linkcolor=blue,
  filecolor=magenta,
  urlcolor=cyan
}
\usepackage{soul}       %% allow wrapping of underlined text, via \ul{...}
  % the following is for biblatex, see
  % https://www.overleaf.com/learn/latex/Biblatex_bibliography_styles
\usepackage[
    backend=biber,
    firstinits=true,
    style=authoryear-ibid,
    maxcitenames=2,
    maxbibnames=25,
    uniquelist=false
]{biblatex}
\usepackage[intoc]{nomencl}  %% for nomenclature
\usepackage{etoolbox} %% for grouping nomenclature, see nomgroup
\usepackage[acronym, toc, nonumberlist]{glossaries} %% for acronyms
\usepackage{geometry}   %% somewhat wider text to allow code
\usepackage[per-mode=symbol]{siunitx}    %% SI Units
\usepackage{amsmath,amssymb,amsthm,bm} %% for math ...
\usepackage{tabularx}
\usepackage{multirow}  %% for tabular
\usepackage{booktabs}  %% toprule, bottomrule
\usepackage{enumitem} %% format description an lists

%
%  MACROS
%  ------------------------------------------------------------------------
%
% a macro to write norms
\newcommand\norm[1]{\lVert#1\rVert}
% a macro to write derivative at:
% examples: 1. $f'(x)\at{x=1}$
%           2. $f'(x)\at[\big]{x=1}$
\newcommand{\at}[2][]{#1|_{#2}}
% Symbols for Love numbers, h, k, and l
\newcommand{\lovek}{k}
\newcommand{\loveh}{h}
\newcommand{\lovel}{l}
% Imaginary i
\newcommand{\iim}{{i\mkern1mu}}
% Quaternion
\DeclareMathAlphabet{\quaternion}{T1}{calligra}{q}{n}

%
% ACRONYMS/GLOSSARY
% ---------------------------------------------------------------------------
%
\makeglossaries
\newacronym{ids}{IDS}{International DORIS Service}
\newacronym{antex}{ANTEX}{Antenna Exchange Format}
\newacronym{pcv}{PCV}{Phase Center Variations}
\newacronym{pco}{PCO}{Phase Center Offset}
\newacronym{iers}{IERS}{International Earth Rotation and Reference Systems Service}
\newacronym{era}{ERA}{Earth Rotation Angle}
\newacronym{gst}{GST}{Greenwich Sidereal Time}
\newacronym{tio}{TIO}{Terrestrial Intermediate Origin}
\newacronym{vlbi}{VLBI}{Very Long Baseline Interferometry}
\newacronym{ode}{ODE}{Ordinary Differential Equation}
\newacronym{rkn}{RKN}{Runge-Kutta-Nystr{\"o}m}
\newacronym{snc}{SNC}{State Noise Compensation}
\newacronym{dmc}{DMC}{Dynamic Model Compensation}
\newacronym{icrf}{ICRF}{International Celestial Reference Frame}
\newacronym{icrs}{ICRS}{International Celestial Reference System}
\newacronym{crs}{CRS}{Celestial Reference System}
\newacronym{gcrf}{GCRF}{Geocentric Celestial Reference Frame}
\newacronym{gcrs}{GCRS}{Geocentric Celestial Reference System}
\newacronym{bcrf}{BCRF}{Barycentric Celestial Reference Frame}
\newacronym{bcrs}{BCRS}{Barycentric Celestial Reference System}
\newacronym{cip}{CIP}{Celestial Intermediate Pole}
\newacronym{cio}{CIO}{Celestial Intermediate Origin}
\newacronym{iau}{IAU}{International Astronomical Union}
\newacronym{cirs}{CIRS}{Celestial Intermediate Reference System}
\newacronym{itrs}{ITRS}{International Terrestrial Reference System}
\newacronym{tirs}{TIRS}{Terrestrial Intermediate Reference System}
\newacronym{itrf}{ITRF}{International Terrestrial Reference Frame}
\newacronym{irf}{IRF}{Inertial Reference Frame}
\newacronym{trf}{TRF}{Terrestrial Reference Frame}
\newacronym{tcb}{TCB}{Barycentric Coordinate Time}
\newacronym{tdb}{TDB}{Barycentric Dynamical Time}
\newacronym{tcg}{TCG}{Geocentric Coordinate Time}
\newacronym{tt}{TT}{Terestrial Time}
\newacronym{tai}{TAI}{International Atomic Time}
\newacronym{eop}{EOP}{Earth Orientation Parameters}
\newacronym{erp}{ERP}{Earth Rotation Parameters}
\newacronym{leo}{LEO}{Low Earth Orbit}
\newacronym{sinex}{SINEX}{Solution INdependent EXchange Format}
\newacronym{cnes}{CNES}{Centre National d'Etudes Spatiales (National Centre for Space Studies)}
\newacronym{catr}{CATR}{Compact Antenna Test Range}
\newacronym{utc}{UTC}{Coordinated Universal Time}
\newacronym{sofa}{SOFA}{Standards Of Fundamental Astronomy}
\newacronym{fcn}{FCN}{Free Core Nutation}
\newacronym{ut1}{UT1}{Universal Time}
\newacronym{tgp}{TGP}{Tide Generating Potential}
\newacronym{pod}{POD}{Precise Orbit Determination}
\newacronym{gmst}{GMST}{Greenwich Mean Sidereal Time}
\newacronym{icgem}{ICGEM}{International Centre for Global Earth Models}
\newacronym{iag}{IAG}{International Association of Geodesy}
\newacronym{igfs}{IGFS}{International Gravity Field Service}
\newacronym{tvg}{TVG}{Time Variable Gravity}
\newacronym{ecef}{ECEF}{Earth Centered Earth Fixed}
\newacronym{grace}{GRACE}{Gravity Recovery and Climate Experiment}
\newacronym{jpl}{JPL}{Jet Propulsion Laboratory}
\newacronym{iugg}{IUGG}{International Union of Geodesy and Geophysics}
\newacronym{jd}{JD}{Julian Date}
\newacronym{mjd}{MJD}{Modified Julian Date}
\newacronym{goce}{GOCE}{Gravity Field and Steady-State Ocean Circulation Explorer}
\newacronym{pece}{PECE}{Predictor-Corrector}
\newacronym{gps}{GPS}{Global Positioning System}
\newacronym{arp}{ARP}{Antenna Reference Point}
\newacronym{ign}{IGN}{Institut Géographique National}
\newacronym{cls}{CLS}{Collecte Localisation Satellites}
\newacronym{grgs}{GRGS}{Groupe de Recherche en Géodésie Spatiale}
\newacronym{doris}{DORIS}{Détermination d'Orbite et Radiopositionnement Intégré par Satellite (Doppler Orbitography and Radiopositioning Integrated by Satellite)}
\newacronym{ggos}{GGOS}{Global Geodetic Observation System}
\newacronym{slr}{SLR}{Satellite Laser Ranging}
\newacronym{gnss}{GNSS}{Global Navigation Satellite System}
\newacronym{igs}{IGS}{International GNSS Service}
\newacronym{ilrs}{ILRS}{International Laser Ranging Service}
\newacronym{ivs}{IVS}{International VLBI Service for Geodesy and Astronomy}
\newacronym{uso}{USO}{Ultra Stable Oscillator}
\newacronym{gsfc}{GSFC}{Goddard Space Flight Center}
\newacronym{nasa}{NASA}{National Aeronautics and Space Administration}
\newacronym{jason}{JASON}{Joint Altimetry Satellite Oceanography Network}
\newacronym{noaa}{NOAA}{National Oceanic and Atmospheric Administration}
\newacronym{eumetsat}{EUMETSAT}{European Organisation for the Exploitation of Meteorological Satellites}
\newacronym{saa}{SAA}{South Atlantic Anomaly}
\newacronym{cddis}{CDDIS}{Crustal Dynamics Data Information System}

%
% Nomenclature
% ---------------------------------------------------------------------------
%
% This code creates the groups, see
% https://www.overleaf.com/learn/latex/Nomenclatures
% -----------------------------------------
\renewcommand\nomgroup[1]{%
  \item[\bfseries
  \ifstrequal{#1}{P}{Physics constants}{%
  \ifstrequal{#1}{V}{Vectors and Matrices}{%
  \ifstrequal{#1}{O}{Other symbols}{}}}%
]}
\nomenclature[P]{$c$}{Speed of light in a vacuum}
\nomenclature[P]{$G$}{Gravitational constant}
\nomenclature[P]{$M_{\Earth}$}{Earth's mass}
\nomenclature[P]{$R_{\Earth}$}{Earth's equatorial radius}
\nomenclature[P]{$\mu _{\Earth}$}{Gravitational constant times Earth's mass}
\nomenclature[P]{$M_{\Sun}$}{Sun's mass}
\nomenclature[P]{$R_{\Sun}$}{Sun's radius}
\nomenclature[P]{$\mu _{\Sun}$}{Gravitational constant times Sun's mass}
\nomenclature[P]{$M_{\Moon}$}{Moon's mass}
\nomenclature[P]{$R_{\Moon}$}{Moon's radius}
\nomenclature[P]{$\mu _{\Moon}$}{Gravitational constant times Moon's mass}

\begin{document}

\title{%
  DSO Geodeticc Library \\
  \large Models, Methods and Software Description}
\author{X. Papanikolaou, D. Anastasiou, V. Zacharis}
\date{\today}

\maketitle
\clearpage

%\frontmatter
\tableofcontents
\listoffigures
\listoftables

%\chapter{Geopotential and Gravity}\label{ch:geopotential_and_gravity}
\section{Third Body Attraction}\label{sec:third_body_attraction}

The acceleration induced (on an orbiting satellite) by celestial bodies 
(denoted by the subscript \texttt{tb}) other than the Earth, are computed 
using a``point mass'' model, i.e.
\begin{equation}\label{eq:thirdbodyacceleration}
  \ddot{\bm{r}} = -GM_{tb} \cdot \left( \frac{\bm{r}-\bm{r}_{tb}}{\norm{\bm{r}-\bm{r}_{tb}}^3} 
    + \frac{\bm{r}_{tb}}{r_{tb}^3} \right)
\end{equation}

where $\bm{r}$ and $\bm{r}_{tb}$ are the geocentric position vectors of the 
satellite and the celestial body respectively, e.g. in \gls{icrf}. Retrieving 
Moon and planetary positions can be perfomed as oulined in \autoref{sec:moonandplanetaryephemeris}.

\subsection{Derivation of Partials and Jacobian}

\textbf{TL;DR} The Jacobian $J=\frac{\partial \ddot{\bm{r}}}{\partial \bm{r}}$ as 
given by \autoref{eq:thirdbodyaccelerationgrad}.

For the Jacobian of \autoref{eq:thirdbodyacceleration}, i.e. 
$\frac{\partial \ddot{\bm{r}}}{\partial\bm{r}}$, we procced for each component 
(assuming $\bm{r}=\begin{pmatrix}x&y&z\end{pmatrix}^{T}$, 
 $\bm{r}_{tb}=\begin{pmatrix}x_{tb}&y_{tb}&z_{tb}\end{pmatrix}^{T}$)

\begin{equation}\label{eq:thirdbodyaccelerationgradx}
  \begin{aligned}
  \frac{\partial \ddot{r}_x}{\partial x} 
    &= \frac{\partial}{\partial x}\left( -GM_{tb} \cdot \left( \frac{x - x_{tb}}{\norm{\bm{r}-\bm{r}_{tb}}^3} + \frac{x_{tb}}{r_{tb}^3} \right) \right) \\
    &= -GM_{tb} \cdot \frac{\partial}{\partial x} \left( \frac{x - x_{tb}}{\norm{\bm{r}-\bm{r}_{tb}}^3} \right) \\
    &= -GM_{tb} \cdot \frac{\partial}{\partial x} \left( \left(x - x_{tb}\right) \cdot \norm{\bm{r}-\bm{r}_{tb}}^{-3} \right) \\
    &= -GM_{tb} \cdot \left[ \frac{\partial}{\partial x} \left(x - x_{tb}\right) \cdot \norm{\bm{r}-\bm{r}_{tb}}^{-3}
     + \left(x - x_{tb}\right) \cdot \frac{\partial}{\partial x} \left( \norm{\bm{r}-\bm{r}_{tb}}^{-3} \right) \right] \\
    &= -GM_{tb} \cdot \left[ \norm{\bm{r}-\bm{r}_{tb}}^{-3} + \left(x - x_{tb}\right) \cdot \frac{\partial}{\partial x} \left( \delta x^2 + \delta y^2 + \delta z^2 \right)^{-3/2} \right] \\
    &= -GM_{tb} \cdot \left[ \norm{\bm{r}-\bm{r}_{tb}}^{-3} + \left(x - x_{tb}\right) \cdot (-3/2) \left( \delta x^2 + \delta y^2 + \delta z^2 \right)^{-5/2} 
      \cdot \frac{\partial}{\partial x} \left( \delta x^2 + \delta y^2 + \delta z^2 \right) \right] \\
    &= -GM_{tb} \cdot \left[ \norm{\bm{r}-\bm{r}_{tb}}^{-3} + \left(x - x_{tb}\right) \cdot \frac{-3}{2} \left( \delta x^2 + \delta y^2 + \delta z^2 \right)^{-5/2} \cdot 2 \delta x \frac{d}{dx} (\delta x) \right] \\
    &= -GM_{tb} \cdot \left[ \frac{1}{\norm{\bm{r}-\bm{r}_{tb}}^{3}} - 3 \frac{\left(x - x_{tb}\right)}{\norm{\bm{r}-\bm{r}_{tb}}^{5}} \cdot \left(x - x_{tb}\right) \right] \\
  \end{aligned}
\end{equation}

The derivatives $\frac{\partial \ddot{r}_y}{\partial y}$ and $\frac{\partial \ddot{r}_z}{\partial z}$ 
can be found in a similar way to be:

\begin{equation}\label{eq:thirdbodyaccelerationgrady}
  \begin{aligned}
  \frac{\partial \ddot{r}_y}{\partial y} &= \frac{\partial}{\partial y}\left( -GM_{tb} \cdot \left( \frac{y - y_{tb}}{\norm{\bm{r}-\bm{r}_{tb}}^3} + \frac{y_{tb}}{r_{tb}^3} \right) \right) \\
    &= -GM_{tb} \cdot \left[ \frac{1}{\norm{\bm{r}-\bm{r}_{tb}}^{3}} - 3 \frac{\left(y - y_{tb}\right)}{\norm{\bm{r}-\bm{r}_{tb}}^{5}} \cdot \left(y - y_{tb}\right) \right]
  \end{aligned}
\end{equation}

and

\begin{equation}\label{eq:thirdbodyaccelerationgradz}
  \begin{aligned}
  \frac{\partial \ddot{r}_z}{\partial z} &= \frac{\partial}{\partial z}\left( -GM_{tb} \cdot \left( \frac{z - z_{tb}}{\norm{\bm{r}-\bm{r}_{tb}}^3} + \frac{z_{tb}}{r_{tb}^3} \right) \right) \\
    &= -GM_{tb} \cdot \left[ \frac{1}{\norm{\bm{r}-\bm{r}_{tb}}^{3}} - 3 \frac{\left(z - z_{tb}\right)}{\norm{\bm{r}-\bm{r}_{tb}}^{5}} \cdot \left(z - z_{tb}\right) \right]
  \end{aligned}
\end{equation}

For the non-diagonal elements of the Jacobian, we have:
\begin{equation}\label{eq:thirdbodyaccelerationgradxy}
  \begin{aligned}
  \frac{\partial \ddot{r}_x}{\partial y}
  &= \frac{\partial}{\partial y}\left( -GM_{tb} \cdot \left( \frac{x - x_{tb}}{\norm{\bm{r}-\bm{r}_{tb}}^3} + \frac{x_{tb}}{r_{tb}^3} \right) \right) \\
  &= -GM_{tb} \cdot \frac{\partial}{\partial y} \left( \left(x - x_{tb}\right) \cdot \norm{\bm{r}-\bm{r}_{tb}}^{-3} \right) \\
  &= -GM_{tb} \cdot \left[ \frac{\partial}{\partial y} \left(x - x_{tb}\right) \cdot \norm{\bm{r}-\bm{r}_{tb}}^{-3} 
    + \left(x - x_{tb}\right) \cdot \frac{\partial}{\partial y} \left( \norm{\bm{r}-\bm{r}_{tb}}^{-3} \right) \right] \\
  &= -GM_{tb} \cdot \left[ \left(x - x_{tb}\right) \cdot \frac{\partial}{\partial y} \left( \delta x^2 + \delta y^2 + \delta z^2 \right)^{-3/2} \right] \\
  &= -GM_{tb} \cdot \left[ \left(x - x_{tb}\right) \cdot (-3/2) \left( \delta x^2 + \delta y^2 + \delta z^2 \right)^{-5/2}
    \cdot \frac{\partial}{\partial y} \left( \delta x^2 + \delta y^2 + \delta z^2 \right) \right] \\
  &= -GM_{tb} \cdot \left[ \left(x - x_{tb}\right) \cdot \frac{-3}{2} \left( \delta x^2 + \delta y^2 + \delta z^2 \right)^{-5/2} \cdot 2 \delta y \frac{d}{dy} (\delta y) \right] \\
  &= -GM_{tb} \cdot \left[ - 3 \frac{\left(x - x_{tb}\right)}{\norm{\bm{r}-\bm{r}_{tb}}^{5}} \cdot \left(y - y_{tb}\right) \right] \\
  \end{aligned}
\end{equation}

which is the same as $\frac{\partial \ddot{r}_y}{\partial x}$. Working in a similar 
way, we can derive that:
\begin{equation}\label{eq:thirdbodyaccelerationgradxz}
  %\begin{aligned}
  \frac{\partial \ddot{r}_x}{\partial z} = \frac{\partial \ddot{r}_z}{\partial x} = 
   -GM_{tb} \cdot \left[ - 3 \frac{\left(x - x_{tb}\right)}{\norm{\bm{r}-\bm{r}_{tb}}^{5}} \cdot \left(z - z_{tb}\right) \right]
  %\end{aligned}
\end{equation}

Putting it all together, we can derive the Jacobian of \autoref{eq:thirdbodyacceleration} 
i.e $J=\frac{\partial \ddot{\bm{r}}}{\partial \bm{r}}$ as: 

\begin{equation}\label{eq:thirdbodyaccelerationgrad}
  \begin{aligned}
  \begin{pmatrix} \frac{\partial \ddot{r}_x}{\partial x} & \frac{\partial \ddot{r}_x}{\partial y} & \frac{\partial \ddot{r}_x}{\partial z} \\
                                                         & \frac{\partial \ddot{r}_y}{\partial y} & \frac{\partial \ddot{r}_y}{\partial z} \\
                                                         &                                        & \frac{\partial \ddot{r}_z}{\partial z} \\
  \end{pmatrix}
  & = -GM_{tb} \cdot [ \bm{I}_{(3\times 3)} \cdot \left( \frac{1}{\norm{\bm{r}-\bm{r}_{tb}}^{3}} \right) \\
  & -\frac{3}{\norm{\bm{r}-\bm{r}_{tb}}^{5}} 
  \begin{pmatrix} \left(x - x_{tb}\right)^2 & \left(x - x_{tb}\right)\left(y - y_{tb}\right) & \left(x - x_{tb}\right)\left(z - z_{tb}\right) \\
                                            & \left(y - y_{tb}\right)^2 & \left(y - y_{tb}\right)\left(z - z_{tb}\right) \\
                                            &                           & \left(z - z_{tb}\right)^2 \\
  \end{pmatrix} ] \\
  &= -GM_{tb} \cdot \left[ \frac{\bm{I}_{(3\times 3)}}{\norm{\bm{r}-\bm{r}_{tb}}^{3}} - 3\frac{\left(\bm{r}-\bm{r}_{tb}\right)\left(\bm{r}-\bm{r}_{tb}\right)^T}{\norm{\bm{r}-\bm{r}_{tb}}^{5}} \right] \\
  \end{aligned}
\end{equation}

\section{Moon and Planetary Ephemeris}\label{sec:moonandplanetaryephemeris}

To retrieve the position at some epoch of the Moon and various planets of the 
solar system, the \href{https://ssd.jpl.nasa.gov/planets/eph\_export.html}{JPL Planetary and Lunar Ephemeris} 
data files are used (so called ``DEXXX''). Additionally, the JPL-provided 
\href{https://naif.jpl.nasa.gov/naif/toolkit.html}{SPICE} software is responsible 
for loading and computing ephemeris.

The SPICE software reads JPL planetary ephemerides in a machine-independent 
binary format (kernels) which are available from the SPICE \href{ftp://ssd.jpl.nasa.gov/pub/eph/planets/bsp}{web site}.
In addition to the ephemeris kernel, one needs to load a ``leapseconds kernel'' 
file, available at \href{https://naif.jpl.nasa.gov/pub/naif/generic\_kernels/lsk/}{https://naif.jpl.nasa.gov/pub/naif/generic\_kernels/lsk/}.
Once both kernels are loaded, then one can compute (interpolate) the positions 
of the Moon or planets at the requested epoch(s).


% for this to work, the following command must be issued:
% makeindex report.nlo -s nomencl.ist -o report.nls
\printnomenclature
\printglossary[type=\acronymtype,title=Acronyms]
\clearpage
\pagenumbering{arabic}

% bibliography
\printbibliography

\end{document}

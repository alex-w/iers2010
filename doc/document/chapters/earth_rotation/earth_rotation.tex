\section{Earth Rotation}

\subsection{Transformation between the International Terrestrial Reference System and the Geocentric Celestial Reference System}
\label{sec:itrs-to-gcrs}

The transformation at any given time $t$ between the \gls{itrs} and \gls{gcrs} 
frames, is given by:
\begin{equation}\label{eq:iers51}
  \bm{r}_{GCRS} = Q(t) R(t) W(t) \bm{r}_{ITRS}
\end{equation}

The implementation defined here, is compliant with the IAU 2000/2006 resolutions and 
follows the so-called ``CIO-based'' procedure (see \cite{iers2010}, Ch. 5).
Note that except if otherwise noted, the parameter $t$ should be given in the 
\gls{tt} time scale.

\subsection{Polar Motion Matrix $W(t)$}\label{ssec:polar-motion-matrix}
The matrix $W$ in \autoref{eq:iers51} accounts for \emph{polar motion}, i.e. relates 
\gls{itrs} and \gls{tirs} and is made up from three (fundamental) rotation matrices:
\begin{equation}\label{eq:iers53}
  W(t) = R_z (-s') R_y (x_{p}) R_x (y_{p})
\end{equation}
where:
\begin{description}
  \item $s'$ is the ``\gls{tio} locator'', necessary to provide an exact realization 
    of the``instantaneous prime meridian'' (i.e. the ``\gls{tio} meridian''). It provides 
    the position of the \gls{tio} on the equator of the \gls{cip}.
  \item $x_p$ is the $x$-component of the \gls{cip} in the \gls{itrs}, and
  \item $y_p$ is the $y$-component of the \gls{cip} in the \gls{itrs}
\end{description}

The formula for $s'$ is given by Eq. (5.13) of \cite{iers2010}.

The last two quantities $(x_p , y_p)$ are known as \emph{polar coordinates} and are 
published by the \gls{iers}. The standard pole coordinates to be used in \autoref{eq:iers53}, 
(if not estimated from the observation) are those published by the \gls{iers} with
additional components to account for the effect of ocean tides and forced terms 
with periods less that two days in space.

\subsection{Implementation}

The library implements the \gls{gcrs} to \gls{itrs} transformation, i.e. :
\begin{equation}
  \bm{r}_{ITRS} = \rotmat{R} \cdot \bm{r}_{GCRS}
\end{equation}

To obtain the inverse transformation matrix, you need to invert the rotation 
matrix $\rotmat{R}$, an operation which in this case is equal to ``transposing'' 
$\rotmat{R}$.

The same transformation can be achieved using a rotation quaternion (see \cite{Bizouard2023}) 
and also provided by the library as option.
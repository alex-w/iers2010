\section{Ocean Tides}\label{sec:ocean-tides}

The dynamical effects of ocean tides are most easily incorporated as periodic variations in 
the normalized Stokes coefficients of degree $n$ and order $m$, as $\Delta C_{nm}$ and $\Delta S_{nm}$. 
Typically, ocean tide models are distributed as gridded maps of tide height amplitudes, providing 
in-phase and quadrature amplitudes of tide heights for selected, main tidal frequencies (or main 
tidal waves), on a variable grid spacing over the oceans (\cite{iers2010}). 

Because ocean tide models came in various formats and flavors, we adopt the description outlined in 
\cite{Torsten23}. In this case, the model grids are converted to spherical harmonic coefficients, 
which are then distributed for each major wave $f$, as \emph{prograde} $(C^{cos}_{nm}, S^{cos}_{nm})_f$ 
and \emph{retrograde} $(C^{sin}_{nm}, S^{sin}_{nm})_f$ potential coefficients. The total effect can 
then be computed as 
\begin{equation}
    \begin{Bmatrix}\Delta C_{nm} \\ \Delta S_{nm}\end{Bmatrix} = 
    \begin{Bmatrix}
        \sum_{j=1}^{f} \left[ \cos{\theta _j} C^{cos}_{nm, j} + \sin{\theta _f} C^{sin}_{nm, j} \right] \\
        \sum_{j=1}^{f} \left[ \cos{\theta _j} S^{cos}_{nm, j} + \sin{\theta _f} S^{sin}_{nm, j} \right]
    \end{Bmatrix}
\end{equation}

where $\theta _f$ is the argument of the tide constituent $f$. Note that in this representation the 
\emph{Doodson-Warburg correction} (see \cite{iers2010}, Ch. 6.6) is already applied (see also 
\cite{Lasser2020} and \cite{Rieser2012}).

To complete the tidal spectrum, admittances between the major tides can be computed using linear 
interpolation. Admittance can be included in the computation via incorporating the relevant files 
distributed by \gls{itsg}.

Available ocean (and atmospheric) tidal models are listed in the relevant \gls{itsg} webpage: 
\url{https://www.tugraz.at/institute/ifg/downloads/ocean-tides} and can be downloaded at 
\url{https://ftp.tugraz.at/outgoing/ITSG/oceanAndAtmosphericTides/models/}. The naming convensions 
and format of the respective files are described in \cite{Torsten23}.

\begin{warning}
    You should set the degree one potential coefficients ($C_{10}$, $C_{11}$ and $S_{11}$) to zero, 
    if the freame of reference is at Center of Mass (CoM).
\end{warning}

\subsection{Implementation}
To create an ocean tide model using the \gls{itsg}-distributed files:
\begin{lstlisting}[language=C++, basicstyle=\footnotesize\ttfamily, frame=single]
    #include "ocean_tide.hpp"

    int degre = ...;
    int order = ...;
    dso::OceanTide octide ("<MODEL>_001fileList.txt", 
        "path/to/<MODEL>_gfc_dir", degree, order);
\end{lstlisting}

\texttt{degree} and \texttt{order} can be omited and will be set to the max degree and 
order read off from the coefficients files.

where \mpath{<MODEL>\_001fileList.txt} is the corresponding \path{001} file and 
\mpath{path/to/<MODEL>\_gfc\_dir} is the path where the prograde and retrograde coefficient files 
are stored.

To include admittance, the constructor should be augmented to hold the \texttt{002} and \texttt{003} 
file, like:
\begin{lstlisting}[language=C++, basicstyle=\footnotesize\ttfamily, frame=single]
    #include "ocean_tide.hpp"

    int degre = ...;
    int order = ...;
    dso::OceanTide octide ("<MODEL>_001fileList.txt", 
        "<MODEL>_002doodson.txt", "<MODEL>__003admittance.txt", 
        "path/to/<MODEL>_gfc_dir", degree, order);
\end{lstlisting}

To compute the potential coefficients summing the effects of all tides (major and admittance if 
available), use:
\begin{lstlisting}[language=C++, basicstyle=\footnotesize\ttfamily, frame=single]
    #include "ocean_tide.hpp"

    int degre = ...;
    int order = ...;
    dso::OceanTide octide ("<MODEL>_001fileList.txt", 
        "<MODEL>_002doodson.txt", "<MODEL>__003admittance.txt", 
        "path/to/<MODEL>_gfc_dir", degree, order);

    octide.stokes_coeffs(tt, ut1(dut1), fargs);
\end{lstlisting}

where \texttt{tt} is the time of computation in \gls{TT}, \texttt{ut1} is the time transformed to 
\gls{UT1} and \texttt{fargs} are the fundamental Delaunay arguments ($l$, $l_p$, $f$, $d$ and 
$\Omega$) computed at the requested epoch.

Example source code is included in the files: 
\mpath{test/costg\_benchmark/check\_11oceanTide\_fes2014b\_34major\_icrf.cpp} 
\mpath{test/costg\_benchmark/check\_11oceanTide\_fes2014b\_with361Admittance\_icrf.cpp}
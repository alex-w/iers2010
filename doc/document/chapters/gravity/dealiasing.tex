\section{Non-Tidal Mass Variations and Dealiasing}\label{sec:non-tidal_mass_variations_and_dealiasing}

Non-tidal temporal variations in the Earth's gravity field can be modelled using the 
well known \gls{aod1b} products (\cite{aod1b07}). These provide a priori information 
about such variations induced by non-tidal circulation processes in atmosphere 
and ocean, computed using \gls{grace} and \gls{gracefo} data. Various releases 
of such products exist (marked as \emph{RLXX} with \emph{XX}=1,2,\dots ,6,7) and 
made available by GFZ (see \url{https://isdc.gfz-potsdam.de/esmdata/aod1b/}). 

Non-tidal \gls{aod1b} files, contain (normalized) Stokes coefficients (i.e. 
$\bar{C}_{nm}$ and $\bar{S}_{nm}$ for spherical harmonic expansion) 
of the anomalous external gravity field of the Earth caused by the mass variability 
predicted from numerical models. Four different coefficients sets are available, 
i.e. 

\begin{displayquote}
the effect of the atmosphere (ATM) that includes the contribution of atmospheric surface 
pressure over the continents; the static contribution of atmospheric pressure to ocean bottom 
pressure elsewhere; and the much weaker contribution of upper-air density anomalies 
above both continents and oceans. Further, the dynamic ocean contribution to ocean 
bottom pressure (OCN) is provided, and the sum of ATM and OCN are given as the so-called 
GLO coefficients, which are typically applied as a background model in precise
orbit determination. For particular oceanographic applications, a fourth set of 
coefficients OBA is additionally included, which is zero over the continents and 
provides the simulated ocean bottom pressure that includes air and water 
contributions elsewhere. Thus, OBA deviates from GLO over the ocean domain only 
by disregarding the small contribution of upper-air density anomalies to 
the external gravity field.
\end{displayquote} \cite{Dobslaw2017}

The temporal resolution is (currently) 3 hours and the max degree/order of the 
coefficients is 180.

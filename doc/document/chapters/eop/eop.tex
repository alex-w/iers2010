\section{\gls{eop}}\label{sec:eop}

\glspl{eop} are needed in a number of places within an orbit determination or 
satellite data analysis proccess. We need efficient ways to:
\begin{enumerate}
  \item retrieve \gls{eop} data files, 
  \item parse and store information, and 
  \item get/interpolate \gls{eop} values for epochs of interest
\end{enumerate}

\subsection{\gls{eop} Products}\label{ssec:eop-products}

Various \gls{eop} product files are made available by a number of different 
sources. Here, we focus on products made available by the \gls{iers}. 

Historically, \gls{iers} provides ``series'' of \gls{eop} products, combining 
different data sets and consistent with different frames. The reference time 
series of \glspl{eop} is called \emph{EOP C04}. This is further categorized in 
\emph{EOP 14C04} aligned with ITRF2014 and ICRF2 (\cite{Bizouard2019}) and 
\emph{EOP 20C04} aligned with ITRF2020 (\cite{iersmail471}). Note that the latter 
follows a slightly different format from the former.

More information can be found on the \gls{iers} website, e.g. 
\href{https://hpiers.obspm.fr/iers/eop/eopc04_14/updateC04.txt}{updateC04.txt} 
and \href{https://hpiers.obspm.fr/iers/eop/eopc04/eopc04.txt}{eopc04.txt}. 
Relevant files can be retrieved from the \gls{iers} website, e.g. 
\href{https://hpiers.obspm.fr/iers/eop/eopc04/eopc04.1962-now}{eopc04.1962-now} for 
\emph{EOP 20C04} and \href{https://hpiers.obspm.fr/iers/eop/eopc04_14/eopc04_IAU2000.62-now}{eopc04\_IAU2000.62-now}.

\subsection{Parsing \gls{eop} Products and Data Structures}\label{ssec:parsing-eop-products-and-data-structures}

At the time of writting, we can parse both \emph{EOP 14C04} and \emph{EOP 20C04} data files. 
The relevant function call, will automatically choose the right format and parse 
\gls{eop} values for a given time range. The values will be stored in a so-called 
\texttt{EopSeries} data structure for further use.

\begin{warning}
\gls{iers}-published \gls{eop} product files contain tabulated values, ``time-stamped'' 
in \gls{utc}. When parsing the files however, the time scale is changed to \gls{tt}.
\end{warning}

Once parsed and stored, one can interpolate \gls{eop} values to any given epoch. 
Interpolation is perfomed using polynomials, via the \emph{Lagrange} methodology. 
There are a couple of things going on, to effectively and accuratelly interpolate 
series of \glspl{eop}:
\begin{itemize}
  \item 
